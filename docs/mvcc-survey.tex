\documentclass{article}
\usepackage{graphicx}
\title{A Quick Survey of MultiVersion Concurrency Algorithms}
\author{Dibyendu Majumdar \\
dibyendy@mazumdar.demon.co.uk \\
Copyright \copyright 2002, 2006}
\date{Revised \today}

\begin{document}

\maketitle

\begin{abstract}
This document looks at some of the problems with traditional methods
of concurrency control via locking, and explains how Multi-Version
Concurrency algorithms help to resolve some of these problems. It
describes the approaches to Multi-Version concurrency and examines a
couple of implementations in greater detail.
\end{abstract}

\section{Introduction}
Database Management Systems must guarantee consistency of data while
allowing multiple transactions to read/write data concurrently.
Classical DBMS \footnote{DBMS implementations that are based upon
the locking protocols of IBM's System R prototype.} implementations
maintain a single version of the data and use locking to manage
concurrency. Examples of SVDB implementations are IBM DB2, Microsoft 
SQL Server, Apache Derby and Sybase.

To understand how Classical DBMS implementations solve concurrency
problems with locking, it is necessary to first understand the type
of problems that can arise.

\subsection{Problems of Concurrency}
\begin{description}
\item[Dirty reads.] The problem of dirty read occurs when one transaction can
read data that has been modified but not yet committed by another
transaction.
\item[Lost updates.] The problem of lost update occurs when one transaction
over-writes the changes made by another transaction, because it does
not realize that the data has changed. For example, if a transaction
T1 reads record R1, following which a second transaction T2 reads
record R1, then the first transaction T1 updates record R1 and
commits the change, following which the second transaction T2
updates R1 and also commits the change. In this situation, the
update made by the first transaction T1 to record R1 is ``lost'',
because the second transaction T2 never sees it.
\item[Non-repeatable reads.] The problem of non-repeatable read occurs when a
transaction finds that a record it read before has been changed by
another transaction.
\item[Phantom reads.] The problem of phantom read occurs when a transaction finds
that the same SQL query returns different set of records at
different times within the context of the transaction.
\end{description}

\subsection{Lock Modes}
To resolve the various concurrency problems, Classical Database
Systems use locking to restrict concurrent access to records by
various transactions. Two types of locks are used:

\begin{description}
\item[Shared Locks.] Are used to protect reads of records. Shared locks
are compatible with other Shared locks but not with Exclusive locks.
Thus multiple transactions may acquire Shared Locks on the same
record, but a transaction wanting to acquire Exclusive Lock must
wait for the Shared locks to be released.
\item[Exclusive Locks.] Are used to protect writes to records. Only
one transaction may hold an Exclusive lock on a record at any time,
and furthermore, Exclusive locks are not compatible with Shared
Locks. Thus a record protected by Exclusive Lock prevents both read
and write by other transactions.
\end{description}

 The various lock
modes used by the DBMS, also called Isolation levels, are given
below:

\begin{description}
\item[Read Committed.] In this mode, the SVDB places Commit duration\footnote{A
Commit duration lock, once acquired, is only released when the
transaction ends.} exclusive locks on any data it writes. Shared locks 
are acquoired on records being read, but these locks are released 
as soon as the read is over. This mode prevents dirty reads, but allows
problems of lost updates, non-repeatable reads, and phantom reads
to occur.
\item[Cursor Stability.] In addition to locks used for Read Committed,
the DBMS retains shared lock on the ``current record'' until 
the cursor moves to another record. This mode prevents dirty reads and lost 
updates.
\item[Repeatable Read.] In addition to exclusive locks used for Read Committed, 
the DBMS places commit duration shared locks on
data items that have been read. This mode prevents the problem of
non-repeatable reads.
\item[Serializable.] In addition to locks used for Repeatable Read, 
when queries are executed with a search
parameter, the DBMS uses key-range locks to lock even non-existent
data that would satisfy the search criteria. For example, if a SQL
query executes a search on cities beginning with the letter 'A', all
cities beginning with 'A' are locked in shared mode even though some
of them may not physically exist in the database. This prevents
other transactions from creating new data items that would satisfy
the search criteria of the query until the transaction running the
query either commits or aborts.
\end{description}

Clearly, the different lock modes offer different levels of data
consistency, trading off performance and concurrency for greater
consistency. Read Committed mode offers greatest concurrency but
least consistency. The Serialized mode offers the most consistent
view of data, but the lowest concurrency due to the long-term locks
held by a transaction operating in this mode.

\section{Problems with Traditional Lock based concurrency}

Regardless of the lock mode used, the problem with a SVDB systems is
that Writers always block Readers. This is because all writes are
protected by commit duration exclusive locks, which prevent Readers
from accessing the data that has been locked. Readers must wait for
Writers to commit or abort their transactions.

In all lock modes other than Read Committed, Readers also block
Writers.

\section{Introduction to Multi-Version Concurrency}

The aim of Multi-Version Concurrency is to avoid the problem of
Writers blocking Readers and vice-versa, by making use of multiple
versions of data.

The problem of Writers blocking Readers can be avoided if Readers
can obtain access to a previous version of the data that is locked
by Writers for modification.

The problem of Readers blocking Writers can be avoided by ensuring
that Readers do not obtain locks on data.

Multi-Version Concurrency allows Readers to operate without
acquiring any locks, by taking advantage of the fact that if a
Writer has updated a particular record, its prior version can be
used by the Reader without waiting for the Writer to Commit or
Abort. In a Multi-version Concurrency solution, Readers do not block
Writers, and vice versa.

While Multi-version concurrency improves database concurrency, its
impact on data consistency is more complex\cite{crit}\cite{si}.

\section{Requirements of Multi-Version Concurrency systems}

As its name implies, multi-version concurrency relies upon multiple
versions of data to achieve higher levels of concurrency. Typically,
a DBMS offering multi-version concurrency (MVDB), needs to provide
the following features:

\begin{enumerate}
\item The DBMS must be able to retrieve older versions of a row.
\item The DBMS must have a mechanism to determine which
version of a row is valid in the context of a transaction. Usually,
the DBMS will only consider a version that was committed prior to
the start of the transaction that is running the query. In order to
determine this, the DBMS must know which transaction created a
particular version of a row, and whether this transaction committed
prior to the starting of the current transaction.
\end{enumerate}

\section{Challenges in implementing a multi-version DBMS}

\begin{enumerate}
\item If multiple versions are stored in the database, an
efficient garbage collection mechanism is required to get rid of old
versions when they are no longer needed.
\item The DBMS must provide efficient access methods that
avoid looking at redundant versions.
\item The DBMS must avoid
expensive lookups when determining the relative commit time of a
transaction.
\end{enumerate}

\section{Approaches to Multi-Version Concurrency}

There are essentially two approaches to multi-version concurrency.
The first approach is to store multiple versions of records in the
database, and garbage collect records when they are no longer
required. This is the approach adopted by PostgreSQL and
Firebird/Interbase.

The second approach is to keep only the latest version of data in
the database, as in SVDB implementations, but \emph{reconstruct}
older versions of data dynamically as required by exploiting
information within the Write Ahead Log. This is the approach taken
by Oracle and MySQL/InnoDb.

The rest of this paper looks at the PostgreSQL and Oracle
implementations of multi-version concurrency in greater detail.

\section{Multi-Version Concurrency in PostgreSQL}

PostgreSQL is the Open Source incarnation of Postgres. Postgres was
developed in University of California, Berkeley, by a team led by
Prof. Michael Stonebraker (of INGRES fame). The original Postgres
implementation offered a multi-version database with garbage
collection. However, it used traditional two-phase locking model
that led to the ``readers blocking writers'' phenomenon.

The original purpose of multiple-versions in the database was to
allow time-travel, and also to avoid the need for a Write-Ahead Log.
However, in PostgreSQL support for time-travel has been dropped, and
the multi-version technology in original Postgres is exploited for
implementing a Multi-Version concurrency algorithm. PostgreSQL team
also added Row level locking and a Write-Ahead Log to the system.

In PostgreSQL, when a row is updated, a new version (called a tuple)
of the row is created and inserted into the table. The previous
version is provided a pointer to the new version. The previous
version is marked ``expired'', but remains in the database until it is
garbage collected.

In order to support multi-versioning, each tuple has additional data
recorded with it:

xmin - The ID of the transaction that inserted/updated the row and
created this tuple.

xmax - The transaction that deleted the row, or created a new
version of this tuple. Initially this field is null.

To track the status of transactions, a special table called
\verb|PG_LOG| is maintained. Since Transaction Ids are implemented
using a monotonically increasing counter, the \verb|PG_LOG| table
can represent transaction status as a bitmap. This table contains
two bits of status information for each transaction; the possible
states are in-progress, committed, or aborted.

PostgreSQL does not undo changes to database rows when a transaction
aborts - it simply marks the transaction as aborted in
\verb|PG_LOG|. A PostgreSQL table therefore may contain data from
aborted transactions.

A Vacuum cleaner process is provided to garbage collect
expired/aborted versions of a row. The Vacuum Cleaner also deletes
index entries associated with tuples that are garbage collected.

Note that in PostgreSQL, indexes do not have versioning information,
therefore, all available versions (tuples) of a row are present in
the indexes. Only by looking at the tuple is it possible to
determine if it is visible to a transaction.

In PostgreSQL, a transaction does not lock data when reading. Each
transaction sees a snapshot of the database as it existed at the
start of the transaction.

To determine which version (tuple) of a row is visible to the
transaction, each transaction is provided with following
information:

\begin{enumerate}
\item A list of all active/uncommitted transactions at the start
of current transaction.
\item The ID of current transaction.
\end{enumerate}

A tuple's visibility is determined as follows (as described by Bruce
Momijian in \cite{bm}):

Visible tuples must have a creation transaction id that:
\begin{itemize}
\item is a committed transaction
\item is less than the transaction's ID and
\item was not in-process at transaction start, ie, ID not in the list of
active transactions
\end{itemize}

Visible tuples must also have an expire transaction id that:
\begin{itemize}
\item is blank or aborted or
\item is greater than the transaction's ID or
\item was in-process at transaction start, ie, ID is in the list of
active transactions
\end{itemize}

In the words of Tom Lane:

A tuple is visible if its xmin is valid and xmax is not. ``Valid''
means ``either committed or the current transaction''.

To avoid consulting the \verb|PG_LOG| table repeatedly, PostgreSQL
also maintains some status flags in the tuple that indicate whether
the tuple is ``known committed'' or ``known aborted''. These status
flags are updated by the first transaction that queries the
\verb|PG_LOG| table.

\section{Multi-version Concurrency in Oracle}

Oracle does not maintain multiple versions of data on permanent
storage. Instead, it recreates older versions of data on the fly as
and when required.

In Oracle, a transaction ID is not a sequential number; instead, it
is a made of a set of numbers that points to the transaction entry
(slot) in a Rollback segment header. A Rollback segment is a special
kind of database table where ``undo'' records are stored while a
transaction is in progress. Multiple transactions may use the same
rollback segment. The header block of the rollback segment is used
as a transaction table. Here the status of a transaction is
maintained, along with its Commit timestamp (called System Change
Number, or SCN, in Oracle).

Rollback segments have the property that new transactions can reuse
storage and transaction slots used by older transactions that have
committed or aborted. The oldest transaction's slot and undo records
are reused when there is no more space in the rollback segment for a
new transaction. This automatic reuse facility enables Oracle to
manage large numbers of transactions using a finite set of rollback
segments. Changes to Rollback segments are logged so that their
contents can be recovered in the event of a system crash.

Oracle records the Transaction ID that inserted or modified a row
within the data page. Rather than storing a transaction ID with each
row in the page, Oracle saves space by maintaining an array of
unique transactions IDs separately within the page, and stores only
the offset of this array with the row.

Along with each transaction ID, Oracle stores a pointer to the last
undo record created by the transaction for the page. The undo
records are chained, so that Oracle can follow the chain of undo
records for a transaction/page, and by applying these to the page,
the effects of the transaction can be completely undone.

Not only are table rows stored in this way, Oracle employs the same
techniques when storing index rows.

The System Change Number (SCN) is incremented when a transaction
commits.

When an Oracle transaction starts, it makes a note of the current
SCN. When reading a table or an index page, Oracle uses the SCN
number to determine if the page contains the effects of transactions
that should not be visible to the current transaction. Only those
committed transactions should be visible whose SCN number is less
than the SCN number noted by the current transaction. Also,
Transactions that have not yet committed should not be visible.
Oracle checks the commit status of a transaction by looking up the
associated Rollback segment header, but, to save time, the first
time a transaction is looked up, its status is recorded in the page
itself to avoid future lookups.

If the page is found to contain the effects of ``invisible''
transactions, then Oracle recreates an older version of the page by
undoing the effects of each such transaction. It scans the undo
records associated with each transaction and applies them to the
page until the effects of those transactions are removed. The new
page created this way is then used to access the tuples within it.

Since Oracle applies this logic to both table and index blocks, it
never sees tuples that are invalid.

Since older versions are not stored in the DBMS, there is no need to
garbage collect data.

Since indexes are also versioned, when scanning a relation using an
index, Oracle does not need to access the row to determine whether
it is valid or not.

In Oracle's approach, reads may be converted to writes because of
updates to the status of a transaction within the page.

Reconstructing an older version of the page is an expensive
operation. However, since Rollback segments are similar to ordinary
tables, Oracle is able to use the Buffer Pool to effectively ensure
that most of the undo data is always kept in memory. In particular,
Rollback segment headers are always in memory and can be accessed
directly. As a result, if the Buffer Pool is large enough, Oracle to
able create older versions of blocks without incurring much disk IO.
Reconstructed versions of a page are also stored in the Buffer Pool.

An issue with Oracle's approach is that if the rollback segments are
not large enough, Oracle may end up reusing the space used by
completed/aborted transactions too quickly. This can mean that the
information required to reconstruct an older version of a block may
not be available. Transactions that fail to reconstruct older
version of data will fail.

\begin{thebibliography}{breitestes Label}
	\bibitem[BM00]{bm} Bruce Momijian. PostgreSQL Internals through Pictures. Dec 2001.
	\bibitem[TL01]{tl} Tom Lane. Transaction Processing in PostgreSQL. Oct 2000.
	\bibitem[MS87]{ms} Michael Stonebraker. The Design of the Postgres Storage System. Proceedings 13th International Conference on Very Large Data Bases (brighton, Sept, 1987). Also, Readings in Database Systems, Third Edition, 1998. Morgan Kaufmann Publishers.
	\bibitem[HB95]{crit} Hal Berenson, Philip A. Bernstein, Jim Gray, Jim Melton, Elizabeth J. O'Neil, Patrick E. O'Neil: A Critique of ANSI SQL
	Isolation Levels. SIGMOD Conference 1995: 1-10.
	\bibitem[AF04]{si} Alan Fekete, Elizabeth J. O'Neil, Patrick E. O'Neil: A Read-Only Transaction Anomaly Under Snapshot Isolation. SIGMOD Record 33(3): 12-14 (2004)
	\bibitem[JG93]{jg} Jim Gray and Andreas Reuter. Chapter 7: Isolation Concepts. Transaction Processing: Concepts and Techniques. Morgan Kaufmann Publishers, 1993.
	\bibitem[ZZ]{zz} Authors unknown. The Postgres Access Methods. Postgres V4.2 distribution.
	\bibitem[DH99]{dh1} Dan Hotka. Oracle8i GIS (Geeky Internal Stuff): Physical Data Storage Internals. OracleProfessional, September, 1999.
	\bibitem[DH00]{dh2} Dan Hotka. Oracle8i GIS (Geeky Internal Stuff): Index Internals. OracleProfessional, November, 2000.
	\bibitem[DH01]{dh3} Dan Hotka. Oracle8i GIS (Geeky Internal Stuff): Rollback Segment Internals. OracleProfessional, May, 2001.
	\bibitem[RB99]{rb} Roger Bamford and Kenneth Jacobs, Oracle.US Patent Number 5,870,758: Method and Apparatus for providing Isolation Levels in a Database System. Feb, 1999.
\end{thebibliography}

\end{document}
